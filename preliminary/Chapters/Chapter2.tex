% Chapter 2

\chapter{Background Research}
\label{Chapter2}

This chapter contains the information found before beginning development of the program, along with some systems that are already available and a summary on them. 

\section{Politics}
Politics is probably the biggest concern when it comes to these bot accounts. They are the reason why false information spreads so fast. This is because of the way Twitter works with its trending hashtags. These bots will tweet and retweet about important and most likely incorrect matters, they also make use of popular hashtags that basically define the topic of a tweet. This then leads to these malicious tags to become trending for everyone to see.

\subsection{2016 US Elections}
The 2016 elections in America was one of the, if not the biggest outburst of Twitter bots we have yet to see. It was found that by extrapolating some findings, roughly 19 percent of 20 million election related tweets originated from bots between September and October of 2016. %https://firstmonday.org/ojs/index.php/fm/article/view/7090/5653
According to the same study it was also found that around 15 percent of all accounts that were involved in election related tweets were bots. Now even though that is a lot of attention for these tweets containing false information, they will mostly only be seen by people who are already on the same side and agree. However, this doesn't rule out the affects. A study by the NBER(National Bureau of Economic Research) came to the conclusion that these bots were the cause of up to 3.23 percent of the votes that went towards Donald Trump. %https://www.nber.org/papers/w24631
This tells us that even if it's just marginal, it does still affect the outcomes.
\paragraph{} The interesting part of all this is that the bots immediately went silent and disappeared as the election ended. The accounts though didn't get deleted but they simply went into hibernation waiting for their next bit of propaganda that needed to be spread. In 2017, 2000 of these bots reemerged to take part in the French and German elections as well, meaning they were run by the same people. They were discovered to make up for 1 in 5 election related tweets. %https://www.motherjones.com/politics/2017/10/twitter-bots-distorted-the-2016-election-including-many-controlled-by-russia/
\subsection{2018 US Mid-Term Election}
%https://www.bbc.co.uk/news/technology-46080157


\section{Existing Systems}
There is a handful of algorithms or programs that have been designed to detect these bots.


\subsection{Tweetbotornot}
Tweetbotornot is a package built in R that uses machine learning to classify Twitter accounts. It has two 'levels'. One for users where it uses information related to an account such as location or number of followers. The other is a tweet-level which checks for details like hashtags, mentions or capital letters out of the user's more recent 100 tweets. 
%https://github.com/mkearney/tweetbotornot
This could prove useful when testing my program to compare results as the accuracy of this library is 93.8 percent. As this is just a package created, it doesn't have any user-interface program built around it or anything like that, therefore is unusable by anyone not knowledgable in R. 


