%Chapter 3

\chapter{System Design}

\label{Chapter3}
This chapter is about the methodology behind the program and some of the features and requirements.

\section{Method}
The core part of the system will be deciding whether the account it is checking is a bot or not. It will use a form of machine learning to classify the account as either 'bot' or 'not bot'. Initially it would seem as if this was a binary classification issue, however it makes more sense to treat it as a regression problem. This makes it much easier to interpret the results for the user as a probability is easily understood and is a much more honest answer compared to just giving the user a 'yes' or 'no' since we can never be too sure either way.


\section{System Requirements}
In order to achieve the aims of the system, there are a few things the program will need to do.
\begin{enumerate}
	\item Allow users to input a Twitter account.
	\item A connection using the Twitter API must be established in order to retrieve users' data.
	\item The system must be able to determine whether an account is a bot or not.
	\item Display the likelihood that an account is a bot or not.
\end{enumerate}

\section{Algorithm}
The algorithm for my system will consist of a supervised machine learning algorithm called random forest. This works by creating a multitude of decision trees and outputting the mean prediction of these trees. However, depending on how things go during development, it might make more sense to use a deep learning neural network instead for the regression. 

\subsection{Data}
Regardless of which algorithm I end up finalising with, I will need data for training. This is important since everything will be based on it. This data needs to also be correctly labelled and will most likely need to be pre-processed somewhat before being fed to my algorithm. It will then be used to train the system.

\subsection{Twitter API}
As the user will be able to enter a Twitter account name themselves, the program will need to have Twitter REST API functionality. This is to access the relevant accounts information such as tweets and account details. There is a limited number of requests allowed within a 15 minute window therefore I need to make sure to keep the API requests to a bare minimum. 

