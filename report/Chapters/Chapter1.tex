% Chapter 1

\chapter{Introduction} % Main chapter title

\label{Chapter1} % For referencing the chapter elsewhere, use \ref{Chapter1} 

Social media has become an integral part of our lives in the past years as we spend more time online than ever before. Roughly 30\% of our online time is spent on social media interactions, Twitter being one of them \cite{globalwebindex}.
Twitter is arguably the largest source of news on the internet. This is due to the nature of information spreading on the platform through tweets and retweets. At this point, because of the scale, it is impossible to monitor it all to make sure everything is accurate and that there is no false information being spread. 

\section{Aim}
The aim of this project is to create a program with an underlying algorithm that will attempt to figure out whether a Twitter account is under the control of a human or is purely being controlled by a script that someone wrote. The issue doesn't come from there being accounts not directly used by people or 'bots'. There are plenty of examples of public bot accounts for things such as weather or news. Instead the issues come with the ones that claim to be real individuals when they in fact are not. With the use of a program such as this, it can be possible to identify these bots by comparing their parameters like tweet content and see if they match patterns of other real people or not.
\newline
Ideally the program shouldn't misjudge too often and should be a reliable way to identify false accounts from real ones. This would be done through a supervised machine learning algorithm which would be fed with as much data of previously labelled accounts as possible.

\section{Overview of report}
The rest of the report is structured as follows;
\newline
Chapter 2 is about background research. It displays some of the areas in which issues arise along with some of the solutions or systems that others have come up with. 
\newline
Chapter 3 contains the system design. This details the method, system requirements and data used during the project. Diagrams are used to display some of these methods. 
\newline
Chapter 4 contains the planning for the project. It describes the processes taken before beginning development to ensure everything would go as planned. 