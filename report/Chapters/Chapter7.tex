

\chapter{Conclusion}

This chapter concludes the report with a conclusion of everything done, what went well and what went wrong. It also includes any possiblities of future work and how this project could be expanded.

\section{System successes}
Overall the system, based on the results, seems to be a success. It has reached most if not all initial requirements and achieved good accuracies on predicting labels. Along with the good accuracies, having decently fast training times is also a success. Another particularly good thing is being able to save an image of a decision tree. This allows us to view it anytime and really get a good idea of what's actually happening within the system.


\section{System failures}
An issue was the neural network model slightly underperforming compared to the other algorithms. This is somewhat made up for by the boost in performance compared to the others.

The data the system relies on for training and testing is a little aged now which could lead to it not predicting as accurately in a live scenario. 


\section{Future work}
There are many things that could be done to further develop this system, here are just a few.

\subsection{Algorithms}
There is always the option to add more machine learning algorithms into the system for comparison. It will provide more options and could potentially get better results.

\subsection{Combining algorithms}
It's possible to build a system that would perhaps combine some mixture of algorithms or layer them, so they work in unison and sort of help each other reach better results. This would be quite a large expansion and would require in-depth knowledge of the algorithms and a lot of research.

\subsection{Expanding features}
Adding more features to the current list would help increase the likelihood that these algorithms find patterns within the data that help them predict the labels correctly. This would require more data with different features.

\subsection{Tweets}
This system could be further developed to not only classify Twitter accounts but also individual tweets. This could lead to finding interesting patterns within tweets, such as tweets made by a bot are more likely to be retweeted by bots then people.

\subsection{Twitter integration}
The ultimate goal would be something like getting this sort of system integrated into Twitter directly. This could lead to bots being removed much more accurately and faster than currently through reports and such.




