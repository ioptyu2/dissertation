% Appendix A

\chapter{Code} % Main appendix title

\label{AppendixA} % For referencing this appendix elsewhere, use \ref{AppendixA}

\section{preprocessing.py}
\begin{lstlisting}
# -*- coding: utf-8 -*-

import numpy as np
import pandas as pd

np.set_printoptions(suppress=True)


def import_data():

    varol_users = np.loadtxt("Z:\ioptyu2\Desktop\gitDissertation\local\Data\\varol_2017.dat")


    col_name = ["UserID","CreatedAt","CollectedAt","Followings","Followers","Tweets","NameLength","BioLength"]
    bot_users = pd.read_csv("Z:/ioptyu2/Desktop/gitDissertation/local/Data/social_honeypot_icwsm_2011/content_polluters.txt",
                                     sep="\t",
                                     names = col_name)


    legit_users = pd.read_csv("Z:/ioptyu2/Desktop/gitDissertation/local/Data/social_honeypot_icwsm_2011/legitimate_users.txt",
                                      sep="\t",
                                      names = col_name)
    
    return [varol_users,bot_users,legit_users,col_name]

#col_name = ["UserID","TweetID","Tweet","CreatedAt"]
#bot_tweets = pd.read_csv("Z:/ioptyu2/Desktop/gitDissertation/local/Data/social_honeypot_icwsm_2011/content_polluters_tweets.txt",
#                                  sep="\t",
#                                  names = col_name)

\end{lstlisting}

\section{random\_forest.py}
\begin{lstlisting}
# -*- coding: utf-8 -*-

from sklearn import datasets
from random import seed
from random import randrange
from math import sqrt
import numpy as np
import preprocessing


iris = datasets.load_iris()

X = iris.data
Y = iris.target
dataset = []
#for i in range(len(X)):
#    dataset.append(np.hstack((X[i],Y[i])))

#cleaning up data(picking out useful rows)
df = preprocessing.import_data()
bots = df[1].values[:2000,3:].astype(int)
legit = df[2].values[:2000,3:].astype(int)


#adding label, bots=1 legit=0
bots = np.hstack((bots,np.ones((bots.shape[0],1))))
legit = np.hstack((legit,np.zeros((legit.shape[0],1))))

dataset = np.vstack((bots,legit))

test_bots = df[1].values[2000:3000,3:].astype(int)
test_legit = df[2].values[2000:3000,3:].astype(int)

test_data = np.vstack((test_bots, test_legit))
test_label = np.vstack((np.ones((test_bots.shape[0],1)), np.zeros((test_legit.shape[0],1))))

#References:
#https://www.python-course.eu/Random_Forests.php
#https://machinelearningmastery.com/implement-random-forest-scratch-python/

#split a dataset into k folds
def cv(dataset,folds):
    dataset_split = list()
    dataset_copy = list(dataset)
    fold_size = len(dataset)/folds
    for i in range(folds):
        fold = list()
        while len(fold) < fold_size:
            index = randrange(len(dataset_copy))
            fold.append(dataset_copy.pop(index))
        dataset_split.append(fold)
    return dataset_split

#calculate accuracy
def accuracy_calc(actual, predicted):
    acc = 0.0
    for i in range(len(actual)):
        if actual[i] == predicted[i]:
            acc += 1.0
    return acc / len(actual) * 100.0

#evaluate the algorithm with cv
def evaluate(dataset, algorithm, folds, *args):
    folds = cv(dataset,folds)
    scores = list()
    for fold in folds:
        train_set = list(folds)
        train_set = sum(train_set, [])
        test_set = list()
        for row in fold:
            row_copy = list(row)
            test_set.append(row_copy)
            row_copy[-1] = None
        predicted = algorithm(train_set, test_set, *args)
        actual = [row[-1] for row in fold]
        accuracy = accuracy_calc(actual, predicted)
        scores.append(accuracy)
    return scores


#split the data into two based on a threshold(value)
def test_split(index, value, dataset):
    left, right = list(), list()
    for row in dataset:
        if row[index] < value:
            left.append(row)
        else:
            right.append(row)
    return left,right

#calculate the gini index
def get_gini(groups, classes):
    instances = float(sum([len(group) for group in groups]))
    gini= 0.0
    for group in groups:
        size = float(len(group))
        if size == 0:
            continue
        score = 0.0
        for class_val in classes:
            p = [row[-1] for row in group].count(class_val) / size
            score += p*p
        gini += (1.0 - score) * (size / instances)
    return gini

#pick the best split 
def best_split(dataset, n_features):
    class_values = list(set(row[-1] for row in dataset))
    b_index, b_value, b_score, b_groups = 999,999,999,None
    features = list()
    while len(features) < n_features:
        index = randrange(len(dataset[0])-1)
        if index not in features:
            features.append(index)
    for index in features:
        for row in dataset:
            groups = test_split(index, row[index], dataset)
            gini = get_gini(groups, class_values)
            if gini < b_score:
                b_index, b_value, b_score, b_groups = index, row[index], gini, groups
    return {'index':b_index, 'value':b_value, 'groups':b_groups}

#create a terminal node value
def terminal(group):
    outcomes = [row[-1] for row in group]
    return max(set(outcomes), key=outcomes.count)

#create child splits for a node or make a terminal
def split(node, max_depth, min_size, n_features, depth):
    left, right = node['groups']
    del(node['groups'])
    #check for no splits
    if not left or not right:
        node['left'] = node['right'] = terminal(left + right)
        return
    #check if max depth
    if depth >= max_depth:
        node['left'], node['right'] = terminal(left), terminal(right)
        return
    #go down left child
    if len(left) <= min_size:
        node['left'] = terminal(left)
    else:
        node['left'] = best_split(left, n_features)
        split(node['left'], max_depth, min_size, n_features, depth + 1)
    #go down right child
    if len(right) <= min_size:
        node['right'] = terminal(right)
    else:
        node['right'] = best_split(right, n_features)
        split(node['right'], max_depth, min_size, n_features, depth + 1)
        

#build a tree
def build_tree(train, max_depth, min_size, n_features):
    root = best_split(train, n_features)
    split(root, max_depth, min_size, n_features, 1)
    return root


#make a prediction using the tree and nodes
def predict(node, row):
    if row[node['index']] < node['value']:
        if isinstance(node['left'], dict):
            return predict(node['left'], row)
        else:
            return node['left']
    else:
        if isinstance(node['right'], dict):
            return predict(node['right'], row)
        else:
            return node['right']


#create a subsample for the data (the bagging/bootstrapping process)
def subsample(dataset, ratio):
    sample = list()
    n_sample = round(len(dataset) * ratio)
    while len(sample) < n_sample:
        index = randrange(len(dataset))
        sample.append(dataset[index])
    return sample

#make a prediction with a list of bagged trees
def bagging_predict(trees, row):
    predictions = [predict(tree, row) for tree in trees]
    return max(set(predictions), key=predictions.count)

#random forest algorithm
def random_forest(train, test, max_depth, min_size, sample_size, n_trees, n_features):
    trees = list()
    for i in range(n_trees):
        #print(i)
        sample = subsample(train, sample_size)
        tree = build_tree(sample, max_depth, min_size, n_features)
        trees.append(tree)
    predictions = [bagging_predict(trees, row) for row in test]
    testing_predictions = [bagging_predict(trees,row) for row in test_data]
    print('Backup testing:', accuracy_calc(test_label, testing_predictions))
    return predictions

#seed
seed(1314)

dataset = np.random.permutation(dataset)


#print results and stuff
n_folds = 2
max_depth  = 10
min_size = 1
sample_size = 1
n_features = int(sqrt(len(dataset[0])-1))
for n_trees in [2,5,10]:
    scores = evaluate(dataset, random_forest, n_folds, max_depth, min_size, sample_size, n_trees, n_features)
    print('Trees: %d' % n_trees)
    print('Scores: %s' % scores)
    print('Mean accuracy: %.3f%%' %(sum(scores)/float(len(scores))))



\end{lstlisting}

\section{sklearnRF.py}
\begin{lstlisting}
# -*- coding: utf-8 -*-
from sklearn.ensemble import RandomForestClassifier
from sklearn.tree import export_graphviz
import pydot
import numpy as np
import preprocessing
from math import sqrt


#1 = bots, 0 = legit
#training data
df = preprocessing.import_data()
train_bots = df[1].values[:12000,3:].astype(int)
train_legit = df[2].values[:12000,3:].astype(int)

feature_list = df[3][3:]

X_train = np.vstack((train_bots,train_legit))

#testing data
test_bots = df[1].values[15000:16000,3:].astype(int)
test_legit = df[2].values[15000:16000,3:].astype(int)

X_test = np.vstack((test_bots,test_legit))

#training labels
train_bots_label = np.ones((train_bots.shape[0],1))
train_legit_label = np.zeros((train_legit.shape[0],1))

Y_train = np.vstack((train_bots_label,train_legit_label))
Y_train = Y_train.ravel(order='C')

#testing labels
test_bots_label = np.ones((test_bots.shape[0],1))
test_legit_label = np.zeros((test_legit.shape[0],1))

Y_test = np.vstack((test_bots_label,test_legit_label))
Y_test = Y_test.ravel(order='C')
a = list()
#random forest algorithm using sklearn
def random_forest(n_trees, max_depth, n_features):
    rf = RandomForestClassifier(n_estimators = n_trees, max_depth = max_depth, max_features = n_features, bootstrap = True)
    rf.fit(X_train, Y_train)
    predictions = rf.predict(X_test)
    save_tree(rf)
    return predictions

#calculate accuracy for predictions
def accuracy(predictions):
#    for i in range(len(predictions)):
#        if predictions[i] < 0.5:
#            predictions[i] = 0.0
#        else:
#            predictions[i] = 1.0
    accuracy = 0.0
    for i in range(len(predictions)):
        if predictions[i] == Y_test[i]:
            accuracy += 1.0
    accuracy = accuracy / len(predictions) * 100
    print("Accuracy: ",accuracy, "%")
    return
#reference for visualisation
#https://towardsdatascience.com/random-forest-in-python-24d0893d51c0

#Visualise a tree
def save_tree(rf):
    tree = rf.estimators_[1]
    export_graphviz(tree, out_file = 'tree.dot', feature_names = feature_list, rounded = True, precision = 1)
    (graph, ) = pydot.graph_from_dot_file('tree.dot')
    graph.write_png('tree.png')
    return


max_depth = 10
n_features = int(sqrt(len(X_test[0])))

for n_trees in [2,5,10,50,100,500,1000]:
    print("Trees: ",n_trees)
    predictions = random_forest(n_trees, max_depth, n_features)
    accuracy(predictions)

\end{lstlisting}

\section{ANN.py}
\begin{lstlisting}
# -*- coding: utf-8 -*-

import numpy as np
import preprocessing
from sklearn.preprocessing import StandardScaler
from keras import layers, models
from keras.utils import to_categorical
import matplotlib.pyplot as plt
from sklearn import datasets



#iris = datasets.load_iris()
#
#X = iris.data
#Y = iris.target
#dataset = []
#for i in range(len(X)):
#    dataset.append(np.hstack((X[i],Y[i])))
#dataset = np.random.permutation(dataset)
#x_train = dataset[:100,:4]
#y_train = dataset[:100,-1]
#
#x_test = dataset[100:120,:4]
#y_test = dataset[100:120,:4]
#
#x_val = dataset[120:,:4]
#y_val = dataset[120:,-1]
#
#y_true = dataset[:,-1]


#############PREPROCESSING##############
df = preprocessing.import_data()
train_bots = df[1].values[:10000,3:].astype(int)
train_legit = df[2].values[:10000,3:].astype(int)

x_train = np.vstack((train_bots,train_legit))


test_bots = df[1].values[12000:16000,3:].astype(int)
test_legit = df[2].values[12000:16000,3:].astype(int)

x_test = np.vstack((test_bots,test_legit))


train_bots_label = np.ones((train_bots.shape[0],1))
train_legit_label = np.zeros((train_legit.shape[0],1))

y_train = np.vstack((train_bots_label,train_legit_label))


test_bots_label = np.ones((test_bots.shape[0],1))
test_legit_label = np.zeros((test_legit.shape[0],1))

y_test = np.vstack((test_bots_label,test_legit_label))
y_true = y_test

val_bots = df[1].values[10000:12000,3:].astype(int)
val_legit = df[2].values[10000:12000,3:].astype(int)

x_val = np.vstack((val_bots, val_legit))

val_bots_label = np.ones((val_bots.shape[0],1))
val_legit_label = np.zeros((val_legit.shape[0],1))

y_val = np.vstack((val_bots_label,val_legit_label))



y_train = to_categorical(y_train)
y_test = to_categorical(y_test)
y_val = to_categorical(y_val)

sc = StandardScaler()
x_train = sc.fit_transform(x_train)
x_test = sc.fit_transform(x_test)
x_val = sc.fit_transform(x_val)

######create model######
model = models.Sequential()
model.add(layers.Dense(64, activation = "relu"))
model.add(layers.Dense(32, activation = 'relu'))
model.add(layers.Dense(2, activation = 'sigmoid'))


model.compile(optimizer = 'adam',
              loss = 'binary_crossentropy',
              metrics = ['accuracy'])

epochs = 100
batch_size = 16

history = model.fit(x_train,
                    y_train,
                    epochs = epochs,
                    batch_size = batch_size,
                    validation_data = (x_val, y_val))


###predictions###
y_pred = model.predict(x_test)

for i in range(len(y_pred)):
    if(y_pred[i][0] < y_pred[i][1]):
        y_pred[i] = 1
    else:
        y_pred[i] = 0
y_pred = y_pred[:,0]

acc = 0.0
for i in range(len(y_pred)):
    if y_pred[i] == y_true[i]:
        acc += 1.0
acc = acc / len(y_pred) * 100
print('Epochs run: ', epochs)
print('Batch size: ', batch_size)
print('Test data size: ', len(y_pred))
print('Accuracy on test data: ', acc, '%')

def comparison_plot(x, 
                    y_A, style_A, label_A, 
                    y_B, style_B, label_B, 
                    title, x_label, y_label):
    
    plt.clf()
    plt.plot(x, y_A, style_A, label = label_A)
    plt.plot(x, y_B, style_B, label = label_B)
    plt.title(title)
    plt.xlabel(x_label)
    plt.ylabel(y_label)
    plt.legend()
    plt.show()

loss = history.history['loss']
val_loss = history.history['val_loss']
comparison_plot(range(1, len(loss) + 1),
                loss, 'bo', 'Training',
                val_loss, 'b', 'Validation',
                'Training and validation loss',
                'Epochs',
                'Loss')

acc = history.history['acc']
val_acc = history.history['val_acc']
comparison_plot(range(1, len(loss) + 1),
                acc, 'bo', 'Training',
                val_acc, 'b', 'Validation',
                'Training and validation accuracy',
                'Epochs',
                'Accuracy')

print("Highest validation accuracy:", val_acc[np.argmax(val_acc)], "Epochs:", np.argmax(val_acc))
print("Lowest validation loss:", val_loss[np.argmin(val_loss)], "Epochs:", np.argmin(val_loss))


\end{lstlisting}