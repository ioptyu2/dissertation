
\chapter{Weekly logs}

\label{AppendixB}

\section{Week1}
Weekly Log (25/01/19)

This was the last week of exams, so I didn’t get much done. Did some more research on what sort of algorithms I should be using and outlining how it is going to work exactly. In the seminar we went over the basics of LaTeX, which is what I will be using to write my reports as it allows a much more structured and customisable approach at writing. I have gone over the template provided to us and have tried to make myself familiar with it. It will take a little getting used to, but it should make everything much more organised. I looked into a bit more about machine learning to see exactly how it will be possible to create this program. 
I had a meeting with my supervisor, where we setup a weekly meeting for Tuesday every week. Since there wasn’t much that I have done this past week, there wasn’t anything to talk about. 
I will be making sure I have the template filled out with all my personal information and make sure it is altered for myself. I also need to make sure to setup a GitHub project as the version control provided will help out a lot.

\section{Week2}
Weekly Log (01/02/19)

This week I was looking to get started on my preliminary report which will contain most of the introductory sections of the report along with plans of how the project will come together. The git repository is now setup and will now be used to store all relevant files on there for easy version control just in case something goes wrong. I have also found this interesting package in R for detecting Twitter bots much like what I plan to create. It is quite interesting to see something like this available online and could be useful in validating results of my own. 
During the meeting with Lahcen, we talked about the preliminary report and what it should contain, as well as completing the compliance statement which was required before getting access to the report page. 
In the seminar we looked at a processing framework called Hadoop. It seems very useful for processing big data and interacting with it in multiple ways. It’s an option that is there in case I need it. 

\section{Week3}
Weekly Log (10/02/19)

This week I spent all my time working on the preliminary report. I worked mainly on the introduction, background research and system design. In the introduction I wrote about the basic idea of the project along with the aims or it. In background research it was about the things I read about before beginning the project and what sort of existing systems there are. As for system design, I wrote about the way the system will work. 
The planning part is still left which I will write this week as well as finish off some sections from the previous ones.
Lahcen said that the work so far on the report is good and I should finish it soon, probably by next week, in order to start looking for data for the actual program. 

\section{Week4}
Weekly Log (17/02/19)

This week was spent adding more stuff to the preliminary report. The sections that needed more in them were planning, system design and background research. 
For planning I needed to create a gantt chart which helps keep track of where I should be. 
Also, after talking with Lahcen, he helped me sort out my referencing which was something that I was lacking. 
Reading week is coming up along with the deadline for the report. Things to do include:
\newline
Finish off background research and system design in report
\newline
Create gantt chart and planning in report
\newline
Find databases to use for system

\section{Week5}
Weekly Log (03/03/19)

I have found some data for the training part of the model. This will all need to be pre-processed as they are mostly in csv format with some unnecessary parts. This was the main task for this week. The other thing Lahcen wanted me to do was to do some more reading about the algorithm (random forest) that I plan to use and to write about it in detail. 
I have also attempted to start with the actual algorithm, although it is quite difficult to begin. 
Next week I will probably need to get a good start on the implementation of random forest. 

\section{Week6}
Weekly Log (10/03/19)

This week I didn’t have much time available to work on the project. Lahcen suggested that I take a break from working on the report and do more things for the program.
I have started working on the implementation of the random forest algorithm. I have found several resources that helped get started along with some pseudocode I found online. About half of the program is done and I think by the end of next week it should be finished to a baseline where I can go back to working on the report for things such as testing. 

\section{Week7}
Weekly Log (17/03/19)

This week I did a mixture of report and programming work. For the report I added parts about the random forest algorithm and used more diagrams to describe the processes such as trees and pseudocode just like Lahcen suggested. 
The algorithm implementation is mostly complete. I’m just waiting for Twitter to still reply to me and send me my API key so I can get started on retrieving data. 
Most of that should be sorted by next week and then I can move on to adding some more finalised sections to the final document and getting some stuff ready to submit for marking over the Easter holiday.

\section{Week8}
Weekly Log (24/03/19)

This week I carried on with the pre-processing that I ended up leaving half finished to work on the report a few weeks ago. The annoying part was to deal with all the hashes in the data of the tweets as this caused issues with python in which hashes are used for comments. I needed to find a way to escape all the characters when importing the dataset. 
I also ended up spending some time creating diagrams. Mainly a UML class diagram and a simple overview of the system to provide aid in describing the system. 
Lahcen mentioned that I needed to swap some sections around. The part about random forest that I wrote needed to be moved to the background research section. 
I need to start thinking about wrapping up the programming part and get started on properly writing towards the final report.
